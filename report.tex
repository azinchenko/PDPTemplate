\documentclass[14pt]{extarticle} %Класс позволяет использовать базовые шрифты бОльших размеров
\usepackage{imit_report}

%=============================
%Персональная настройка макета
%=============================
%!!!
%Здесь могут располагаться дополнительные команды тонкой настройки


%=============================
%Конец Персональная настройка макета
%=============================

%Подключение литературы
\addbibresource{literature.bib}

\begin{document}
	
%%%--------Титульная страница
%==Титульная страница
\thispagestyle{empty}
\begin{center}
Министерство науки и высшего образования Российской Федерации\\
Федеральное государственное бюджетное образовательное\\
учреждение высшего образования\\
<<Иркутский государственный университет>>\\
(ФГБОУ ВО <<ИГУ>>)\\
Институт математики и информационных технологий\\
Кафедра алгебраических и информационных систем\\
\end{center}

\vspace{3.7cm}

\begin{center}
{\bf 
	ОТЧЕТ\\[1mm]
	ПО ПРЕДДИПЛОМНОЙ ПРАКТИКЕ
	\\[1mm]

}  

\end{center}

\vspace{1.8cm}

{
\noindent\hbox to 0.48\textwidth {%
	\mbox{ } \hfil} %
	\begin{tabular}[t]{l}
		Студент 4 курса очного отделения\\
		Группа 02461--ДБ\\
		Фамилия Имя Отчество
	\end{tabular}		
}

\vspace{1.3cm}

{
\noindent\hbox to 0.48\textwidth {%	
	\mbox{ } \hfil} %
	\begin{tabular}[t]{l}
		Руководитель:\\ степень, должность\\
		Фамилия И.О.		
	\end{tabular}		
}

\vspace{1.3cm}
{
\noindent\hbox to 0.48\textwidth {%	
	\mbox{ } \hfil} %
	\begin{tabular}[t]{l}
		Начало практики: 24.04.2023 г.\\
		Окончание практики: 04.06.2023 г.		
	\end{tabular}		
}



\vfill 
\noindent
\begin{minipage}{\textwidth}
\centering	\bf Иркутск 2023
\end{minipage}

\newpage
%%%----------------------Содержимое отчета%%%
\renewcommand{\baselinestretch}{1.5}
\normalsize
\nocite{*} %Команда подключает весь список литературы без ссылок
%-------------
%Основная часть
%-------------
	
\myreportsection{Задачи, поставленные на преддипломную практику}

\begin{enumerate}
	
    \item Перечисляете задачи, поставленные на преддипломную практику.
    \item ...
    \item Количество задач может быть разное. Последняя задача у всех одинаковая.
	\item Завершить оформление ВКР. 
	
\end{enumerate}
	
\myreportsection{Отчет по выполнению поставленных задач}

%\textit{Ниже указываются задачи и результаты выполнения поставленных задач}

\mytasksection{Задача 1. Название задачи 1}

Краткое описание по ее выполнению (объем описания можно смотреть в предоставленных примерах)

\mytasksection{Задача 2. Название задачи 2}

Краткое описание по ее выполнению


\mytasksection{Задача 6. Название задачи 6}

Краткое описание по ее выполнению

\mytasksection{Задача 7. Завершить оформление ВКР}

За время прохождения преддипломной практики было завершено оформление выпускной квалификационной работы. 

Данная выпускная квалификационная работа изложена на \makebox[5mm]{\hrulefill} страницах. Она состоит из введения, трёх глав, заключения, списка использованных источников и \makebox[5mm]{\hrulefill} приложений. Список литературы состоит из \makebox[5mm]{\hrulefill} наименований. В работе присутствуют \makebox[5mm]{\hrulefill} листингов кода, \makebox[5mm]{\hrulefill} таблиц и \makebox[5mm]{\hrulefill} рисунков.

{\it Далее говорите, о чем идет речь в каждой главе и даете краткое описание каждого параграфа. Например:}

Первая глава содержит обзор предметной области.

В параграфе 1.1 рассказывается об основных принципах растрового представления информации, цветовых моделях и координатах, приводится классификация алгоритмов, которые выполняются над растровыми изображениями. Какие алгоритмы для чего выполняются. Рассказано, почему в работе выбраны те или иные алгоритмы.

В параграфе 1.2 приводится обзор сервиса, демонстрирующего принцип работы рассматриваемых алгоритмов, для изучения методов, лежащих в основе алгоритмов. Рассказывается, почему был найден только один такой сервис.

...

В заключении подводятся итоги проделанной работы, перечисляются результаты решения задач, поставленных на выпускную квалификационную работу.

%Для включения списка литературы подключается файл literature.bib
\renewcommand{\refname}{Список использованных в  выпускной квалификационной работе источников}
\printbibliography
\nopagebreak
\vspace{1cm}
Дата 05 июня 2023\hspace{6cm}Подпись

\end{document}

